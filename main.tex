%!TEX TS-program = xelatex
\documentclass[fontsize=14pt, twoside]{scrreprt}
\usepackage{leadsheets}
\newcommand\secTOC[1]{
    \subsection*{#1}
    \addcontentsline{toc}{section}{#1}
    \markboth{#1}{#1}}
 \newcommand\subsecTOC[1]{
    \subsection*{#1}
    \addcontentsline{toc}{subsection}{#1}
    \markboth{#1}{#1}}
\newcommand\chapTOC[1]{
    \chapter*{#1}
    \addcontentsline{toc}{chapter}{#1}
    \markboth{#1}{#1}}
\definesongtitletemplate{mytemp}{\subsecTOC{\songproperty{title}} \songproperty{subtitle}}}
\setleadsheets{title-template = mytemp}
\usepackage{polyglossia} %XeLaTeX
\usepackage{fontspec-xetex}
\setmainfont{Linux Biolinum O}
\usepackage{xunicode}
\setdefaultlanguage{english} %XeLaTeX
\setotherlanguage{german} 
%\usepackage[T1]{fontenc}
%\usepackage[utf8]{inputenc} %PDFLaTeX
%\usepackage[british]{babel}
\usepackage{float}
\usepackage{multicol}
\usepackage[a4paper, tmargin=1cm, lmargin=1cm, rmargin=1cm]{geometry}
\usepackage{setspace}
\usepackage{caption}
\usepackage{rotating}
\usepackage{fbox}
\usepackage{hyperref}
\usepackage[version=4]{mhchem}
\usepackage{graphicx}
\usepackage{amsmath,amsthm,amssymb,bm}
\usepackage{multirow}
\usepackage{gensymb}
\usepackage{acronym}
\usepackage{subcaption}
\setlength{\columnseprule}{1pt}
\setlength{\parindent}{0cm}
% \setstretch{1}
\title{Lieder Die Wir Singen}
\author{Capoeira Equilíbrio}
\date{\today}
\graphicspath{{bilder/}}
\begin{document}
\maketitle
\newpage
\tableofcontents
% \newpage
\chapter*{Notation}
\textbf{Sao bento Grande}: \chord{I}Ching,  \chord{II}Ching , \chord{III}Ding, \chord{IV}Dong, \chord{V}Dong

\textbf{Angola}: \chord{I}Ching,  \chord{II}Ching , \chord{III}Dong, \chord{IV}Ding, \chord{V}Chack!

\textbf{Pandeiro, Atabaque (etc)}: \chord{III}Bum, \chord{IV}Pa, \chord{V}Bum
\newpage

% \section{A}
% \newpage
\begin{multicols*}{2}
\chapTOC{A}
\begin{song}{title={A Capoeira tem que ter molejo}}
\begin{verse*}
A Capoeira tem que ter molejo \\
A Capoeira tem que ter molejo \\
a Capoeira tem que balanḉar,\\
a Capoeira tem que ter magia,\\
tem que ter energia pra poder jogar\\

    \textit{A Capoeira tem que ter molejo \\
    A Capoeira tem que ter molejo \\
    a Capoeira tem que balanḉar, \\
    a Capoeira tem que ter magia, \\
    tem que ter energia pra poder jogar \\}
\end{verse*}
\end{song}

\begin{song}{title={A Bananeira caiu}}
\begin{verse*}

O facao bateu embaixo\\
\textit{A bananeira caiu} \\
Cai cai caiu bananeira \\
\textit{An bananeira caiu} \\
\end{verse*}
\end{song}
\columnbreak
\begin{song}{title={A hora é essa}}
    \begin{verse*}
        a hora é essa \\
        a hora é essa \\
        \textit{
            a hora é essa \\
            a hora é essa \\
        }
    \end{verse*}
\begin{verse*}
 \chord*{I}be rimbau tocou na capoeira \\
berimbau tocou eu vou jogar \\
\textit{berimbau tocou na capoeira \\ 
berimbau tocou eu vou jogar}
\end{verse*}

% zweite strophe?(moeno canaeeee sukandubilaaoooo)

\end{song}

\begin{song}{title={Abalou Capoeira}}
\begin{verse*}
Abalou Capoeira abalou\\
O abalou deixa abalar\\
\textit{Abalou Capoeira abalou}\\
O abalou deixa abalar\\
\textit{Abalou Capoeira abalou}\\

\chord*{I}A balou Capoeira abalou\\
\chord*{III}Mais \ se abalou deixa abalar\\
\chord*{I}\textit{A} \textit{balou Capoeira abalou}\\
Mais se abalou deixa abalar\\
\textit{Abalou Capoeira abalou}\\

\end{verse*}
\end{song}
\begin{song}
    
\end{song}
\begin{song}{title={Adeus povo bom adeus}}
    \begin{verse*}
         \chord*{IV}A deus povo bom adeus\\
        Adeus que eu ja vou me embora\\
        Pelas ondas do mar eu vim\\
        Pelas ondas do mar eu vou me embora\\
    \end{verse*}
        \end{song}
\pagebreak
\begin{song}{title={Ai Ai Aidê}}
    \begin{verse*}
     \chord*{III}Ai, Ai, Aidê\\
    \chord*{IV}Aidê, Aidê, Aidê, Aidê\\
    \chord*{III}\textit{Ai}  \ \textit{Ai, Aidê}\\
    \chord*{III}Joga \ bonito que eu quero ver\\
    \textit{Ai, Ai, Aidê}\\
    Aidê, Aidê, Aidê, Aidê\\
    \textit{Ai, Ai, Aidê}\\
    Joga bonito que eu quero aprender\\
    \textit{Ai, Ai, Aidê}\\
    Aidê, Aidê, Aidê, Aidê\\
    \textit{Ai, Ai, Aidê}\\
    Joga bonito pra mim, pra vocé\\
    \textit{Ai, Ai, Aidê}\\
    \end{verse*}
    \end{song}

\begin{song}{title={Agora sim}}

\begin{verse*}
Agora sim que mataram\\
seu besouro\\
depois de morto bisorinho\\
cordão de ouro\\
\textit{Agora sim, que mataram}\\
\textit{seu besouro}\\
\textit{depois de morto bisorinho}\\
\textit{cordão de ouro}\\
\end{verse*}

\begin{verse*}
\chord*{I}Bi sorinho é um\\
\chord*{I}\textit{Cor} \textit{dão de ouro}\\
\end{verse*}

\end{song}

\begin{song}{title={Amanha e dia de santos (um, dois três)}}

    \begin{verse*}
        \chord*{V}A manhã é dia santo\\
         \textit{(Um, dois, três)}\\
        Dia de povo de deus\\
\textit{Três e três: seis}\\
Quem tem roupa vai na missa\\
\textit{Seis e três: nove}\\
Quem não tem faz como eu\\
\textit{Nove e três: doze}\\

Um dois três\\
Três e três: seis\\
Seis e três: nove\\
Nove e três: doze\\
\textit{Um dois três}\\
\textit{Três e três: seis}\\
\textit{Seis e três: nove}\\
\textit{Nove e três: doze}\\


\end{verse*}
\end{song}

\begin{song}{title={Apanha a Laranja}}

    \begin{verse*}
        Apanha a laranja no chão tico-tico\\
        Se meu amor foi embora, eu não fico\\

        \textit{Apanha a laranja} \textit{no chão tico-tico}\\
        Meu abada é de renda, è de fita\\
        \textit{Apanha a laranja} \textit{no chão tico-tico}\\
        Nao pegue com mao só com pé o com bico\\
        \textit{Apanha a laranja} \textit{no chão tico-tico} \\  
    \end{verse*}

\end{song}



\columnbreak
\begin{song}{title={Apanha a laranja menino}}

    \begin{verse*}
        Apanha laranja menino\\
        Apanha laranja no chão\\
        Defenda o seu reino sozinho\\
        Do fundo do seu coração\\

        \textit{Apanha a laranja menino...}\\
        Meu papagaio vou\\
        Da enchente da maré\\
        \textit{Meu papagaio voou}\\
        \textit{Da enchente da maré}\\
        Corro paco papaco papaco\\
        Meu loro não sabe o que quer\\
        \textit{Corro paco papaco papaco}\\
        \textit{Meu loro não sabe o que quer}\\
    \end{verse*}
\end{song}


\newpage
\chapTOC{B}


\begin{song}{title={Bahía Axé}}
    \begin{verse*}
        Quem bom estar com você\\
        \textit{Estar com você}\\
        Aqui nessa roda\\
        \textit{Aqui nessa roda},\\
        Com, esse conjunto\\
        \textit{Com esse conjunto}\\
        Axé, Capoeira, \\
        \textit{Capoeira axé}\\
        \textbf{oder} \\
        Axé pro meu mestre\\
        \textit{Pro meu mestre axé},\\

        Axé, Equilibrio \\
        \textit{Equilibrio Axé} \\ 
        \textbf{oder} \\
        O vento, que parte tão lindo\\
        \textit{Que, parte, tão lindo}\\
        Em cima dos coqueiras\\
        Que vem de lá de traz\\
        Meditereraneo (\textit{Mediteraneo})\\
        Io, io io io io io\\
        Io io io io io\\
        No brasil tem axe Do Brasil vem axé\\
        \textit{Io, io io io io io}\\
        \textit{Io io io io io}\\
        \textit{Em Bonn tem axe Do Brasil vem axé}\\

    \end{verse*}
\end{song}


\columnbreak
\begin{song}{title={Balança Equilíbrio}}
    \begin{verse*}
        \chord*{IV}Ba lança Equilíbrio\\
        Balança eu quero ver\\
        Balança Equilíbrio\\
        \chord*{III}Ca poeira pra valer\\

        \begin{chorus*}
        \textit{Balança Equilibrio}\\
        \textit{Balança eu quero ver}\\
        \textit{Balança Equilibrio}\\
        \chord*{III}\textit{Ca} \textit{poeira pra valer}\\
        \end{chorus*}
        
        Eu venho lá de Santos\\
        Vim aqui pra te mostrar\\
        O estilo que é santista\\
        e é de se admirar\\
        \textbf{(Balança Equilibrio...)}\\

        Mestre Índio vem da Terra\\
        que o consagrou\\
        Capoeira Equilíbrio\\
        estila de grande valor\\
        \textbf{(Balança Equilíbrio...)}\\

        De Santos a volta mundo\\
        Com seu mestre já rodou\\
        Se dedique meu aluno\\
        Quero te ver professor\\
        \textbf{(Balança Equilíbrio...)}\\

        Capoeira é uma arte\\
        de muita dedicação\\
        agradeço a meu mestre\\
        por ter dado a liçao\\
    \end{verse*}
\end{song}

\begin{song}{title={Beira mar ioiô}}
    \begin{verse*}
        \chord*{III}Bei ra mar ioiô\\
        Beira mar iaiá\\
        \textit{Beira mar ioiô}\\
        \textit{Beira mar iaiá}\\

        \chord*{I}Bei ra mar beira mar\\
        \chord*{I}\textit{é} \ \textit{ de ioiô}\\
        beira mar, beira mar\\
        \textit{é de iaiá}\\
    \end{verse*}
\end{song}

\begin{song}{title={Balança que Pesa Ouro}}
        \begin{verse*}
        Balança que Pesa ouro \\ 
        Não é pra pesar metal \\ 
        Tem passarinho pequeno \\ 
        Que mata cobra coral \\
                
        \textit{Balança que Pesa ouro \\ 
        Não é pra pesar metal \\ 
        Tem passarinho pequeno \\ 
        Que mata cobra coral} \\
        \end{verse*}
\end{song}

\columnbreak
\begin{song}{title={Berimbau faz (Chin Chin Dong Dong)}}
        \begin{verse*}
            \chord*{V}Dong, \ Dong, Dong\\
            Berimbau que faz\\
            Dsching Dsching Dong Dong\\
            \textit{Dong, Dong, Dong\\
            Berimbau que faz\\
            Dsching Dsching Dong Dong}\\
            \chord*{I}Be rimbau Faz \\
            (Chor unterbricht den Sänger hier)\\
            \chord*{I}\textit{Dschin} \ \textit{Dsching Dong Dong}\\
            Berimbau Chama\\
            \textit{Dschin Dsching Dong Dong}\\
            Berimbau XXX\\
            \textit{Dschin Dsching Dong Dong}\\
        \end{verse*}
\end{song}

\columnbreak
 \begin{song}{title={Biriba é pau é Madeira}}
         \begin{verse*}
             Madeira de maḉaranduba\\
             Madeira de jacarandá\\
             \textit{Madeira de maḉaranduba}\\
             \textit{Madeira de jacarandá}\\
             Beriba é pau, é Madeira\\
             \textit{Beriba é pra tocar}\\
         \end{verse*}
 \end{song}
 

 \begin{song}{title={Boa viagem}}
     \begin{verse*}
     Adeus, adeus\\
     \textit{Boa viagem!}\\
     Eu já embora \\
     \textit{Boa viagem!}\\
     \textbf{oder}\\
     Izzy vai embora\\
     \textit{Boa viagem!}\\
     \textbf{oder}\\
     Pendeiro vai embora...\\
     Eu vou com Deus\\
     \textit{Boa viagem!}\\
     E com Nossa Senhora \\
     \textit{Boa viagem!}\\
     Adeus \\
     \textit{Boa viagem\!}\\
     \end{verse*}
 \end{song}
 %
 \begin{song}{title={Bota fogo no mato}}
         \begin{verse*}
             \chord*{I}Bo ta fogo no mato\\
             chama ele que ele vem\\
             Bota fogo no mato\\
             chama ele que ele vem\\
             \textit{Bota fogo no mato}\\
             \textit{chama ele que ele vem}\\
         \end{verse*}
 \end{song}
 \chapTOC{C}


\begin{song}{title={Camugere}}
        \begin{verse*}
            Camugerê como vai como ta\\
            \textit{Camugerê}\\
            Como vai vosmucê\\
            \textit{Camugerê}\\
            Como ta de saúde\\
            \textit{Camugerê}\\
            Muito dem, meu prazer\\
            \textit{Camugerê}\\
        \end{verse*}
\end{song}


\columnbreak

\begin{song}{title={Canarinho da alemanha}}
    \begin{chorus*}
            \chord*{I}Ca narinho da alemanha\\
            Quem matou meu curio\\
            Canarinho da alemanha\\
            Quem matou meo curio\\
            \textit{Canarinho da alemanha}\\
            \textit{Quem matou meu curio}
    \end{chorus*}
        \begin{verse*}
            Eu jogo capoeira\\
            mas sei meu mestre é melhor\\
            \textit{Canarinho da alemanha...}\\
            \textit{Quem matou meu curio}\\
            Eu jogo capoeira\\
            Na Bahia e Maceió\\
            \textit{Canarinho da alemanha.}\\
            O segredo da lua\\
            Quem sabe clarao do sol\\
            \textit{Canarinho da alemanha...}\\
            Na road a Capoeira,\\
            quero ver quem é melhor\\
            \textit{Canarinho da alemanha...}\\
            Eu jogo Capoeira\\
            mas seu Pastinha é melhor\\
            \textit{Canarinho da alemanha...}\\
        \end{verse*}
\end{song}
\columnbreak
\begin{song}{title={Capiniero}}
        \begin{verse*}
            \chord*{IV}Ca piniero de ioio\\
            \chord*{IV}Ca piniero de iaia\\
            \chord*{IV}\textit{Ca}piniero de ioio\\
            \chord*{IV}\textit{Ca} \ \textit{piniero de iaia}\\
            \chord*{IV}\textit{Não}  \textit{corta capinia eee}\chord*{I}\textit{e} \\
            \textit{Só corta quando eu mand}\chord*{I}\textit{ar} \\
            Nao mexe com o mar imbondo \\
            Só mexe quando eu mandar\\
            \chord*{IV}Ol ha a cana da moenda \chord*{III}\textit{pra} \ fazer garapa\\
            \textit{Olha a cana da moenda pra fazer garapa}\\
            \chord*{IV}Ol ha o milho do cabão \chord*{III}pra fazer fubar\\
            \textit{Olha o milho do cabão pra fazer fubar}\\
            \chord*{I}Eu \ andava sete léguas \chord*{III}sem olhar \chord*{III}pra \ atrás\\
            \textit{Eu andava sete léguas sem olhar pra atrás}\\
            \chord*{III}Ca da passo que eu dava era um passo amais\\
            \textit{Cada passo que eu dava era um passo amais}\\
        \end{verse*}
\end{song}

\columnbreak
\begin{song}{title={Capoeira é da nossa cor}}
        \begin{verse*}
            au ê au ê au ê ê\\
            E Lê lê lê lê lê lê lê lê lê ô\\
            \textit{au ê au ê au ê ê}\\
            \textit{E Lê lê lê lê lê lê lê lê lê ô}\\
            tá no sangue da raça brasileira\\
            Capoeira\\
            \textit{é da nossa cor}\\
            Berimbau\\
            \textit{é da nossa cor}\\
            Atabaque\\
            \textit{é da nossa cor}\\
            O pandeiro\\
            \textit{é da nossa cor}\\
            au ê au ê au ê ê\\
            E Lê lê lê lê lê lê lê lê lê ô\\
            \textit{au ê au ê au ê ê}\\
            \textit{E Lê lê lê lê lê lê lê lê lê ô}\\
        \end{verse*}
\end{song}

\begin{song}{title={Casa nova}}
        \begin{verse*}
            Ave Maria meu deus\\
            Nunca vi casa nova cai\\
            Nunca vi casa nova cai\\
            Nunca ci casa velha cai\\
            \textit{Ave Maria meu deus}\\
            \textit{Nunca vi casa nova cai}\\
            Nunca vi casa nova cai\\
            Nunca vi Mestre Sombra cai\\
            \textit{Ave Maria meu deus}\\
            \textit{Nunca vi casa nova cai}\\
        \end{verse*}
\end{song}


\chapTOC{D}

\begin{song}{title={Dendê o Dendê}, subtitle={\begin{small}
        \url{https://www.lalaue.com/capoeira-music/dende-o-dende-angola/}
\end{small}}}
        \begin{verse*}
            \chord*{IV}Den dê O Dendê\\
            Dendê O Dendê\\
            Dendê é de Agnola\\
            Angola é de Dendê\\
            \textit{Dendê O Dendê}\\
            \textit{Dendê O Dendê}\\
            \textit{Dendê é de Angola}\\
            \textit{Angola é de Dendê}\\
            Mestre Pastin\chord*{I}ha \\
            \chord*{III}\textit{Foi} \ \textit{embora}\\
            Mestre Jao Pequeno\\
            \textit{Foi embora}\\
            Mestre Liminha\\
            \textit{Foi embora}\\
        \end{verse*}
\end{song}

\columnbreak
\begin{song}{title={Dificuldade no nó do atabaque eu vou fazer}}
        \begin{verse*}
            \chord*{III, V}Difi \chord*{IV}cul dade no nó do atabaque eu vou fazer\\
            \chord*{III}Ba lança pra la\\
            \chord*{III}Ba lança pra ca\\
            \chord*{III}pra \ tu aprender\\
            \textit{Dificuldade no nó do atabaque eu vou fazer}\\
            \textit{Balança pra la}\\
            \textit{balança pra ca}\\
            \textit{pra tu aprender}\\
            \chord*{III}Tem \ dendê no atabaque\\
            \chord*{III}\textit{No} \textit{atabaque tem dendê}\\
            \chord*{III}a tabaque tem dendê\\
            \chord*{III}\textit{Tem} \ \textit{dendê no atabaque}\\
        \end{verse*}
\end{song}

\columnbreak
\begin{song}{title={Din Din}}
        \begin{verse*}
            \chord*{IV}Din Din é o nome de vovo\\
            Din Din é o nome de madrinha\\
            Din Din Din Din Din Din e dinheiro dinheiro for feito pra gastar\\
            \textit{Din Din eu o nome de vovo}\\
            \textit{Din Din e o nome de madrinha}\\
            \textit{Din Din Din Din Din Din e dinheiro}\\
            \textit{dinheiro for feito pra gastar}\\
            A minha vôo é mãe da minha mãe\\
            A minha mãe é irmã da minha madrinha\\
            Se a minha mãe não der dinheiro\\
            Eu peso pra minha dindinha\\
            Dinheiro foi feito para gastar\\
            Amor Amizade e Capoeira\\
            Nos temos que preservar\\
        \end{verse*}
\end{song}


\begin{song}{title={Dois, dois, dois}}
        \begin{verse*}
            \chord*{V}Do is dois dois\\
            Dois dois um\\
            Capoeira de Angola\\
            Capoeira zum zum zum\\
            \textit{Dois dois dois}\\
            \textit{dois dois um}\\
            \textit{Capoeira de Angola}\\
            \textit{Capoeira zum zum zum}\\
            Sete e sete são quartoze três vezes sete vonte e um\\
            Quem souber bem soletra as paixões de cada um\\
            Dois dois\\
        \end{verse*}
\end{song}


\begin{song}{title={Dona Maria como vai você}}
        \begin{verse*}
            E vai você vai você?\\
            \textit{Dona Maria como vai você?}\\
            E pega na vassoura como vai você\\
            \textit{Dona Maria como vai você?}\\
            E mais joga bonito que eu quero ver\\
            \textit{Dona Maria como vai você?}\\
            E vai você vai voce?\\
            \textit{Dona Maria como vai você?}\\
            O jogo bonito que eu quero aprender\\
            \textit{Dona Maria como vai você?}\\
            Como vai você como vai você\\
            \textit{Dona Maria como vai você?}\\
            Como tá voce eu quero caber\\
            \textit{Dona Maria como vai você?}\\
            Como tá como passou como vai voce?\\
            \textit{Dona Maria como vai você?}\\
            Canta bonito que quero você\\
            \textit{Dona Maria como vai você?}\\
            Olha jogo ligeiro que eu quero ver\\
            \textit{Dona Maria como vai você?}\\
        \end{verse*}
\end{song}

\begin{song}{title={Dona Maria ia ia}}
        \begin{verse*}
            Dona maria ia ia\\
            É do Camboatá io io\\
            Ela chega na venda iaia\\
            Ela manda botar\\
        \end{verse*}
\end{song}

\chapTOC{E}


\begin{song}{title={Ê Capoeira Ê}, subtitle={Mestre Índio}}
    \begin{chorus*}
        \textit{Ê Capoeira Ê \\
        Ea vamos vadiar\\}
    \end{chorus*}
        \begin{verse*}
            Minha Capoeira não nasceu la na Bahia mas ela eu sei respeitar.\\
            Meu Mestre vem de Santos\\
            E sabe representar Mestre Índio \\ 
            \textbf{oder}\\
            Mestre Eduardo\\
            \textbf{oder}\\
            Mestra Kátia\\
            \textbf{oder}\\
            Mestre Sombra\\

            Vamos vadiar camarada\\
            Vamos vadiar camara\\
        \end{verse*}
\end{song}



\begin{song}{title={E e e viola}}
        \begin{verse*}
            E e e viola\\
            E e e Violatoca que eu quero jogar viola\\
        \end{verse*}
\end{song}

\columnbreak
\begin{song}{title={Em cada Som Em cada Toque}}
    \begin{chorus*}
            Laaaa la awê la awê la awê la awa\\
            La La awê la awa\\
            \textit{Laaaa la awê la awê la awê la awa}\\
            \textit{La La awê la awa}\\
    \end{chorus*}
        \begin{verse*}

            Que grupo\\
            Maravilhosa esse\\
            È o grupo equilibrio\\
            Que eu am hoje\\ 
            \textbf{(La awê...?)}\\
            Em cada som em cada toque em cada ginga\\
            um estilo de jogo\\
            \textit{Em cada som em cada toque\\
            em cada ginga}\\
            \textit{um estilo de jogo}\\
            La la awê la awê la awê la awa\\
            \textit{Laaaa la awê la awê la awê la awa}\\
            \textit{La La awê la awa}\\
        \end{verse*}
\end{song}

\columnbreak
\begin{song}{title={Eu fui No Sonho}}
    \begin{chorus*}
            \textit{Eu fui No sonho\\
            No Barracão de Waldémar\\
            Eu fui No sonho\\
            No Barracão de Waldémar\\}
    \end{chorus*}
    \begin{verse*}

            Eu fui No sonho\\
            No Barracão de Waldémar\\
            Eu fui No sonho\\
            No Barracão de Waldémar\\

            quando cheguei no portão\\
            vi Besouro Mangangá\\
            protejendo o barração\\
            pra maldade não entrar\\
\\
            a roda de outros tempos\\
            com violão e berimbau\\
            calçado ou descalço\\
            Angola ou Regional\\
\\
            vadiando na Angola\\
            vi Noronha e seu Pastinha\\
            vi Bimba e Maré\\
            toquando Santa Maria\\

            eu vi Caribé pintando\\
            sentado pelo chão\\
            imortalizando\\
            momentos do barração\\

            na função do Barracão\\
            tinha bamba e sopeiro\\
            e a forma de cantar\\
            era gritar no terreiro\\

            logo quando acordei\\
            tive o berimbau na mão\\
            toquei e homenagiei\\
            Seu Waldemar da Paixãoent\\
    \end{verse*}
\end{song}

\begin{song}{title={Eu vim pra lhe ver}}
       \begin{verse*}
           \chord*{IV}Eu \ vim pra lhe ver\\
            Pra saber da sua saúde.\\
            Meu cavalo em paco na ladeira\\
            \chord*{III}eu \ vim pra lhe ver mais não pude.\\
       \end{verse*} 
\end{song}

\columnbreak
\begin{song}{title={Eu vim aqui buscar}}
        \begin{verse*}
            \chord*{IV}Eu \ vim aqui buscar\\
            \chord*{III}Um pouquinho de dendê\\
            \textit{Eu vim aqui buscar}\\
            \textit{ um pouquinho de dendé}\\
            \chord*{III}Prá \  passar no atabaque \\
            (no berimbau na viola no\\
            mestre indio)\\
            \chord*{III}\textit{Um}\textit{ pouqinho de dendé}\\
            \chord*{IV}Mo rena Morena me da\\
            \chord*{III}\textit{Um}\textit{ pouqinho de dendé}\\
        \end{verse*}
\end{song}

\begin{song}{title={Eu vi fazer xue xua }}
        \begin{verse*}
            \chord*{III}Eu pisei na folha seca\\
            vi fazer xue xua\\
            \chord*{III}Xu e xue xue xua\\
            \chord*{I}\textit{Eu} \textit{vi fazer xue xua}\\
            Xue xue xue xua\\
            \textit{Eu vi fazer xue xua}\\
            \chord*{III}Jo gue as pernas para cima\\
            \chord*{III}De ixa o corpo vadiar\\
            \chord*{III}Xu e xue xue xua\\
            \chord*{I}\textit{Eu} \textit{vi fazer xue xua}\\
        \end{verse*}
\end{song}

\begin{song}{title={Eu vou botar a minha rede na varanda}}
        \begin{verse*}
            \chord*{I}E u vou botar a minha rede na varanda\\
            Eu quero ver a minha rede balançar\\
            Balança rede iôiô (Pause für Kanon)\\
            Balança rede iâiâ (Pause für Kanon)\\
        \end{verse*}
\end{song}

\chapTOC{F}

\begin{song}{title={Foi na Beira do Mar}}
        \begin{verse*}
            \chord*{I}Fo i na beira do mar\\
            Foi na beira do mar\\
            aprendei a jogar Capoeira de Angola\\
            do na beira do mar\\

            \chord*{I}\textit{Foi} na beira do mar\\
             \textit{Foi na beira do mar\\}
            \textit{aprendei a jogar Capoeira de Angola do na beira do mar\\}
        \end{verse*}
\end{song}

\columnbreak
\begin{song}{title={Foi No Clarão da Lua}}
    \begin{chorus*}
            \textit{Foi\\
            Foi no clarão da Lua\\
            Que eu vi acontecer\\
            Num vale tudo com o Jiu-Jitsu\\
            O capoeira vencer\\}
    \end{chorus*}
        \begin{verse*}
            Deu armada deu rasteira\\
            meia-Lua e a ponteira\\
            Logo no primeiro round
            venceu o capoeira\\
            Embaixo do ringue\\
            Mestre Mentirinha vibrava\\
            Tocando seu berimbau\\
            Enquanto a Muzenza cantava Mas foi\\
            \textbf{(Foi...)}\\
            Deu armada deu rasteira \\
            meia-Lua e a ponteira\\
            Logo no primeiro round\\
            venceu o capoeira\\
            Embaixo do ringue \\ 
            Mestre Mentirinha vibrava\\
            Tocando seu berimbau\\
            Enquanto a Muzenza cantava Mas foi\\
            \textbf{(Foi...)}
        \end{verse*}
\end{song}


\chapTOC{G}

\chapTOC{H}

\begin{song}{title={Hoje Tem capoeira}}
        \begin{verse*}
            \chord*{I}Ho je tem Capoeira\\
            Hoje e dia de Roda\\
            O dendê ta bom\\
            o dendê ta bom\\
            \chord*{V}Sente o dendê\\
        \end{verse*}
\end{song}

\chapTOC{I}

\chapTOC{J}

\begin{song}{title={Jogo de dentro jogo de fora}}
        \begin{verse*}
            Jogo de dentro jogo de fora\\
            Jogo bonito esse jogo de Angola\\
            \textit{Jogo de dentro jogo de fora}\\
            Jogo manhaso esse jogo de Angola\\
            \textit{Jogo de dentro jogo de fora}\\
            Jogo bonito berimbau e viola\\
            \textit{Jogo de dentro jogo de fora}\\
            Jogo bonito quero ver agora\\
            \textit{Jogo de dentro jogo de fora}\\
        \end{verse*}
\end{song}

\chapTOC{K}

\chapTOC{L}

\begin{song}{title={La Vai Viola}}
    \begin{verse*}
        O lê lê la vai viola\\
        \textit{Tim Tim Tim la vai viola}\\
        O viola meu bem viola\\
        \textit{Tim Tim Tim la vai viola}\\
        Jogo o bonito no jogo de angola\\
        \textit{Tim Tim Tim la vai viola}\\
        Jogo de dentro e jogo de fora\\
        \textit{Tim Tim Tim la vai viola}\\
        \end{verse*}
\end{song}

\begin{song}{title={Leee La Lae Lae La}}
        \begin{verse*}
            Lee la Lae Lae La\\
            La Lae Lae La\\
            Le Le Le Le La La\\
            \textit{La lae lae la}\\
\\
            Coro\\
            Berimbau chamou pro jogo\\
            Pandeiro me respondeu\\
            O atabaque já entrou\\
            Mestre Bimba apareceu\\
            \textit{La lae lae la}\\
\\
            Coro\\
            Manoel dos Reis Machado\\
            Criador de Regional\\
            Espalhando pelo mundo\\
            Essa cultura national\\
            \textit{La lae lae la}\\

            \
            Coro\\
            Capoeira començou\\
            Como roda tradiçional\\
            Era luta e defesa\\
            Do negro no carnavial\\
            \textit{La lae lae la}\\
\\
            Coro\\
            Manuel dos Reis Machado\\
            estivador da beira do cais\\
            incorporpu jogo de angola\\
            com batuque e mutio mais\\
            \textit{La lae lae la}\\
\\
            Coro\\
            Lá no cais se batizou\\
            A Capoeira Reginal\\
            Espalhando pelo mundo\\
            Essa arte marcial\\
            \textit{La lae lae la}\\
\\
            Coro\\
            Lee la Lae Lae La\\
            La Lae Lae La\\
            Le Le Le Le La La\\
            \textit{La lae lae la}\\
        \end{verse*}
\end{song}


\begin{song}{title={Lá em casa tem um cidadão}}
        \begin{verse*}
            Lá em casa tem um cidadão\\
            quando eu passo por ele \\ 
            um aperto de mão\\
            Lá em casa tem um bacharel quando\\
            eu passo ele tira o chapéu!\\
        \end{verse*}
\end{song}
\columnbreak
\begin{song}{title={Leva eu}}
        \begin{verse*}
            Leva \chord*{V}e u leva eu pra vadiar (aaaa)\\
            \chord*{III}To ca \chord*{V}e u \berimbau com dendê (eeee)\\
            \chord*{V}Le va eu pra vadiar\\
            \chord*{V}To ca o berimbau a (eeee)\\
        \end{verse*}
\end{song}

\begin{song}{title={Luanda e meu Pandeiro}}
        \begin{verse*}
            Luanda e meu Pandeiro\\
            Luanda e meu Para\\
            \textit{Luanda e meu Pandeiro}\\
            \textit{Luanda e meu Para}\\
            Oi Tereza samba deitada\\
            Oh Marina sambe de pe\\
            Oh la na cais da Bahia\\
            Na roda de Capoeira\\
            nao tem lelê nem lalá\\
            ôlalaê la ê la\\
            \textit{ô lê lê}\\
            O la la ê\\
            o la la ê la la ê la\\
            O la la ê la la ê la la ê la\\
            O lele\\
            \textit{La la ê}\\
            \textit{o la la ê la la ê la}\\
        \end{verse*}
\end{song}

\chapTOC{M}

\begin{song}{title={Mae e Mae}}
        \begin{verse*}
            \chord*{III}Ma e e mae\\
            \chord*{III}Pa e e pae\\
            \chord*{V}So u xodo de mamae\\
            \chord*{V}So u xodo de papae\\
            \textit{Mae e mae}\\
            \textit{pae e pae}\\
            \textit{sou xodo de mamae}\\
            \textit{sou xodo de papae}\\
            \chord*{III}Eu sou\\
            \chord*{III}\textit{xo}  \textit{do de mamae}\\
            eu sou\\
            \textit{xodo de papae}\\
        \end{verse*}
\end{song}

\begin{song}{title={Mandei caiá meu Sobrado}}
    \begin{verse*}
            Mandei caiá meu sobrado \\
            Mandei, mandei, mandei\\
            Mandei pintar de amarelo\\
            Caiei, Caiei, Caiei\\
    \end{verse*}
        
\end{song}
\begin{song}{title={Mare me Leva e Mare me traz}}
        \begin{verse*}
            \chord*{I}Ma re me Leva e Mare me traz\\
            \textit{Mare me leva e mare me traz}\\
        \end{verse*}
\end{song}

\begin{song}{title={Maria Conga}}
        \begin{verse*}
            \chord*{I}Ma ria Conga\\
            \chord*{V}Es se quilombo é nosso\\
            \chord*{V}Es se quilombo é seu é meu\\
            \chord*{V}Es se quilombo é nosso.\\
            \textit{Maria Conga}\\
            \textit{Esse quilombo é nosso}\\
            \textit{Esse quilombo é seu é meu}\\
            \textit{Esse quilombo é nosso.}\\
        \end{verse*}
\end{song}

\columnbreak
\begin{song}{title={Meu Barco Minha Canoa}}
        \begin{verse*}
            \chord*{IV}Me u barco minha canoa\\
            Não paro de remar\\
            Eu remo a favor da maré\\
            \chord*{III}Nã o deixo o meu barco virar\\

            \textit{Meu barco minha canoa }\\
            \textit{não paro de remar}\\
            \textit{Eu remo a favor da maré}\\
            \textit{Não deixo o meu barco virar}\\
            \\
            Meu barco minha canoa\\
            não paro de remar\\
            Eu remo a favor da maré\\
            Não deixo essa fonte secar\\
            \\
            \textit{Meu barco minha canoa }\\
            \textit{não paro de remar}\\
            \textit{Eu remo a favor da maré}\\
            \textit{Não deixo essa fonte secar}\\

            \chord*{III}Não deixo o meu barco virar\\
            \textit{Não deixo o meu barco virar}\\
            Não deixo essa fonte secar!\\
            \textit{Não deixo essa fonte secar}\\
        \end{verse*}
\end{song}
\columnbreak
\begin{song}{title={Meu Cantador}}
        \begin{verse*}
            \chord*{I}O \ meu cantador\\
            como vai você e e\\
            Venho visita iaia\\
            Vem aqui pra lhe vem\\
            Aus internet:\\
            ê cantador como vai você eee\\
            Vim lhe visitar ia ia é vim aqui lhe ver\\
            ê cantador como vai você eee\\
            Vim lhe visitar ia ia é vim aqui lhe ver\\
            êeee cantador quando é\\
            ê nos conhece no olhar\\
            ê na mandinga do jogo aia\\
            ê e tambem no jogar\\
            ê cantador como vai você eee\\
            Vim lhe visitar ia ia é vim aqui lhe ver\\
            ê cantador quando é bom\\
            ê não precisa chamar\\
            ele sabe sua hora aia\\
            ê de cantar e de versar ia ia\\
            ê cantador como vai você eee\\
            Vim lhe visitar ia ia é vim aqui lhe ver\\
        \end{verse*}
\end{song}

\columnbreak
\begin{song}{title={Minha Veranda}}
        \begin{verse*}
            \chord*{V}Aqui é minha casa\\
            minha varanda meu dendê\\
            \textit{Aqui é minha casa}\\
            \textit{minha varanda meu dendê}\\

            \chord*{V}Meu chapéu de palha\\
            minha massapê\\
            \textit{Meu chapéu de palha}\\
            \textit{minha massapê}\\

            Cada um tem sua história\\
            É bom respeitar\\
            Pra conquistar minha varanda\\
            Não foi fácil camará\\
            \textit{Aqui é minha casa}\\
            \textit{minha varanda meu dendê}\\

            Se não sabe a minha história\\
            entre aqui vou te contar\\
            Tem berimbau pandeiro\\
            atabaque pra tocar\\
            Na parede um quadro\\
            de Carybé mandei pintar\\
            E minha varanda\\
            é de frente pro mar\\
            \textit{Aqui é minha casa}\\
            \textit{minha varanda meu dendê}\\
        \end{verse*}
\end{song}

\chapTOC{N}


\begin{song}{title={Na beira mar do barra fora}}
        \begin{verse*}
            \chord*{I}N a beira mar do barra fora\\
            Vadiando com Aidê\\
            Lamento com a minha viola\\
            Angola que tem dende\\
        \end{verse*}
\end{song}

\begin{song}{title={Nem tudo que reluz é ouro}}
        \begin{verse*}
            Nem tudo que reluz é ouro\\
            Nem tudo que balanḉa cai\\
            \textit{(Nem tudo que reluz é ouro}\\
            \textit{Nem tudo que balanḉa cai)}\\
            Cai cai cai cai\\
            Capoeira balanḉa mas nao cai\\
            \textit{(Cai cai cai cai)}\\
            Batuqueiro balanḉa mas nao cai\\
            \textit{(Cai cai cai cai)}\\
            Foi você que me disse\\
            que filho de Bimba nao cai\\
            \textit{(Foi você que me disse}\\
            \textit{que filho de Bimba nao cai)}\\
        \end{verse*}
\end{song}

\begin{song}{title={Nhem, Nhem, Nhem}}
        \begin{verse*}
            \chord*{III}Nh em Nhem Nhem\\
            \chord*{I}O menino chorou\\
            \chord*{III}\textit{Nh} \textit{em nhem nhem}\\
            \chord*{I}E menino chorou\\
            \chord*{III}\textit{Nh} \textit{em nhem nhem}\\
            \chord*{Po}rque não mamou\\
            \chord*{III}\textit{Nh} \textit{em nhem nhem}\\
            Sua mãe tá na feira\\
            \textit{Nhem nhem nhem}\\
            Ela ainda, nao voltou\\
            \textit{Nhem nhem nhem}\\
            Cala boca menino\\
            \textit{Nhem nhem nhem}\\
            O menino é danado\\
            \textit{Nhem nhem nhem}\\
            O menino é malvado\\
            \textit{Nhem nhem nhem}\\
            O menino chorou\\
            \textit{Nhem nhem nhem}\\
            Chorou chorou\\
            \textit{Nhem nhem nhem}\\
        \end{verse*}
\end{song}

\begin{song}{title={No Mar}}
        \begin{verse*}
            No mar\\
            No mar no mar no mar\\
            No mar eu vi cantar\\
            No mar no mar no mar\\
            Minha sereia ela e sereia\\
            Menina dos olhos verdes\\
            Que foi para o mar navegar\\
            Traiga a minha jangada\\
            Que ficou la no fundo do mar\\
            Puxa rede marinheiro\\
            Puxa rede puxa\\
            Foi chamar marinheiro\\
            Foi chamar\\
            Eu sou de tempo\\
            Sou de Ogum\\
            Na beira mar\\
            Aue meu cana meninha\\
            Mas eu sou aue\\
            Meu canavial\\
            Cana que nao da acucar\\
            Ay menina\\
            Nao fica no canavial\\
        \end{verse*}
\end{song}

\begin{song}{title={Noite sem lua}, subtitle={\begin{small}
\end{small}}}
       \begin{verse*}
            Era uma noite sem lua.\\
            Era uma noite sem lua.\\
            Era uma noite sem lua\\
            \textit{Era uma noite sem lua.\\
            Era uma noite sem lua.\\
            Era uma noite sem lua a a a}\\

            Ô, era uma noite sem lua e eu tava sozinho.\\
            Fazendo do meu caminhar o meu próprio caminho.\\
            Sentindo o aroma das rosas e a dor dos espinhos.\\

            \textit{
            De repente, apesar do escuro, olhe eu pude saber\\
            Que havia alguém me espreitando, sem ter nem porquê\\
            Era hora de luta de morte, é matar ou morrer}\\

            A navalha passou me cortando, era quase um carinho\\
            O meu sangue misturou-se ao pó e às pedras do caminho\\
            Era hora de pedir Axé para o meu orixá\\
            E partir para o jogo da morte, é perder ou ganhar.\\

            \textit{Coro}\\
            Eu dei o bote certeiro da cobra, alguém me guiou\\
            Meia-Lua bem dada é a morte\\
            E a luta acabou.\\

            \textit{Coro}\\
            Eu segui pela noite sem Lua,\\
            Histórias na algibeira\\
            Não é fácil acabar com a sorte de um bom Capoeira\\
            
            \textit{Coro}\\
            Se você não acredita, me espera num outro caminho\\
            E prepara bem a sua navalha, eu não ando sozinho\\
       \end{verse*}
\end{song}

\chapTOC{O}

\begin{song}{title={O ABC}}
        \begin{verse*}
            \chord*{IV}E o A,\\
            e o B\\
            E o A, E o B, E o C\\
            \textit{(E o A,}\\
            \textit{e o B}\\
            \textit{E o A, E o B, E o C)}\\
            \chord*{I}Ai ai ai ai\\
            \chord*{I}Eu nao sei ler\\
            \textit{(Ai ai ai ai)}\\
            O ABC\\
            \textit{(Ai ai ai ai)}\\
            Eu nao sei ler\\
            \textit{(Ai ai ai ai)}\\
            O ABC\\
            \chord*{IV}E o A, no atabaque, \\ 
            E o B, no Berimbau\\ 
            e o C na Capoeira\\
            \textit{E o A, no atabaque,\\
            E o B no Berimbau \\ 
            E o C na Capoeira}\\
        \end{verse*}
\end{song}

\columnbreak

\begin{song}{title={O barco passou no rio}}
        \begin{verse*}
            O barco passou no rio \\
            e fez ondas que nem no mar\\
            \textit{O barco passou no rio \\
            e fez ondas que nem no mar}\\
            Eu tava vadiando nao vi o barco passar\\
            \textit{O barco passou no rio\\
            e fez ondas que nem no mar}\\
            Cuidado canoeiro pra canoa nao virar\\
            \textit{O barco passou no rio \\
            e fez ondas que nem no mar}\\
            Cuidado angoleiro pra onda nao te levar\\
            \textit{O barco passou no rio\\
            e fez ondas que nem no mar}\\
        \end{verse*}
\end{song}


\begin{song}{title={O pé pela mao}}
        \begin{verse*}
            Vieram três pra bater no nego\\
            \textit{Vieram três pra bater no nego}\\
            Trouxeram faca porrete e facao\\
            \textit{Trouxeram faca porrete e facão}\\
            Voce nao sabe o que pode fazer o nego\\
            \textit{Voce nao sabe o que pode fazer o nego}\\
            Troca as maos pelo pé\\
            \textit{E o pé pela mao}\\
            Troca a mao pelo pé\\
            \textit{E o pé pela mao}\\
            Troca as maos pelo pé\\
            \textit{E o pé pela mao}\\
            Troca pé pela mao\\
            \textit{E a mao pelo pé}\\
            Voce nao sabe o que pode fazer o nego\\
            \textit{Voce nao sabe o que pode fazer o nego}\\
            Tapa na cara rasteira no chao\\
            \textit{Tapa na cara rasteira no chão}\\
            Troca as maos pelo pé\\
            \textit{E o pé pela mao}\\
            Troca a mao pelo pé\\
            \textit{E o pé pela mao}\\
            Troca as maos pelo pé\\
            \textit{E o pé pela mao}\\
            Troca pé pela mao\\
            \textit{E a mao pelo pé}\\
        \end{verse*}
\end{song}


\begin{song}{title={Oh, areia}}
        \begin{chorus*}
            \chord*{III}O h, areia!\\
            Oh areia (do mar)!\\
            \textit{Oh areia!}\\
            \textit{Oh areia!}\\
        \end{chorus*}
        \begin{verse*}
            \chord*{I}A bre os meus caminhos \textit{Areia!}\\
            Eu quero passar \textit{Areia!}\\
            Sou pequininho \textit{Areia!}\\
            Mas eu chego lá \textit{Areia!}\\
            \textbf{(Ref.)}\\
            \\
            Areia do rio \textit{Areia!}\\
            Areia do mar \textit{Areia!}\\
            Sou pequeninho \textit{Areia!}\\
            Deus me guiar \textit{Areia!}\\
        \end{verse*}
\end{song}



\begin{song}{title={Oi Sim Sim}}
        \begin{verse*}
            \chord*{III}O i, Sim Sim Sim\\
            Oi, Nao Nao Nao\\
            \textit{Oi, Sim Sim sim}\\
            \textit{Oi nao nao nao}\\
            Mas hoje tem, amanha nao\\
            Mas hoje tem, amanha nao\\
            \textit{Oi, Sim Sim sim...}\\
            Mas hoje tem, amanha nao\\
            Olha a pisada de Lampio\\
            \textit{Oi, Sim Sim sim...}\\
        \end{verse*}
\end{song}

\begin{song}{title={Olha noite de lua}}
        \begin{verse*}
            \chord*{IV}Ol ha noite de lua eee\\
            Olha noite de lua aaa\\
        \end{verse*}
\end{song}

\chapTOC{P}

\begin{song}{title={Parabens a voce}}
        \begin{verse*}
            Parabéns a você,\\
            nesta data querida,\\
            muito felicidade,\\
            muitos anos de vida\\
            O(a) nome faz anos,\\
            O arat é só dele(a)\\
            Cada ano que passa,\\
            Ele(a) fica mais velho(a)\\
            Abalou Capoeira abalou\\
            \chord*{I}A balou Capoeira abalou\\
            \chord*{III}Ma is se abalou deixa abalar\\
            \chord*{I}A \textit{balou Capoeira abalou}\\
            Mais se abalou deixa abalar\\
            \textit{Abalou Capoeira abalou}\\
        \end{verse*}
\end{song}

\columnbreak
\begin{song}{title={Paraná e}}
        \begin{verse*}
            Paraná e, Paraná e, Parana\\
            \textit{Paraná e, Paraná e, Parana}\\
            Que dirá minha mulher\\
            Capoeira me venceu, Paraná\\
            \textit{Paraná e, Paraná e, Parana}\\
            Minha Mae é mulher velha, Paraná\\
            Fecha porta e dorme cedo, Paraná\\
            \textit{Paraná e, Paraná e, Parana}\\
            A mulher para ser bonita, Paraná\\
            Nao precisa se pintar, Paraná\\
            \textit{Paraná e, Paraná e, Parana}\\
            A pintura é do Diabo, Paraná\\
            A beleza é Deus quem dá, Paraná\\
            \textit{Paraná e, Paraná e, Parana}\\
            Sete e seta sao quatorze, Paraná\\
            e mais sete é vinte-um, Paraná\\
            \textit{Paraná e, Paraná e, Parana}\\
            Quem nao sabe com mandinga, Paraná\\
            nao carrega patuá, Paraná\\
            \textit{Paraná e, Paraná e, Parana}\\
        \end{verse*}
\end{song}

\begin{song}{title={Pega meu canario}}
        \begin{verse*}
            Pega meu canarío\\
            Pega na caveat\\
            meu canrío canta\\
            Xo borboleta\\
            \textit{Pega meu canarío}\\
            \textit{Pega na caveat}\\
            \textit{meu canrío canta}\\
            \textit{Xo borboleta}\\
            Eu nau vou na sua casa\\
            pra você nao vim na minha\\
            você tem a boca grande\\
            Vai comer minha galinha\\
            \textit{Pega meu canarío...}\\
        \end{verse*}
\end{song}

\begin{song}{title={Peito vazio}}
        \begin{verse*}
            Sinto um vazio no peito\\
            Berimmbau vem me ajudar\\
            Vem vem vem\\
            berimbau vem me ajudar\\
            \textit{(Sinto um vazio no peito}\\
            \textit{Berimbau vem me ajudar}\\
            \textit{Vem vem vem}\\
            \textit{berimbau vem me ajudar)}\\
            Eu sinto saudades de um tempo\\
            Que o berimbau me levou\\
            Agora eu levee le para\\
            os lugares one eu vou\\
            \textit{Sinto um vazio no peito...}\\
            Existem milhoes de estrellas\\
            Mas a minha eu encontrei\\
            Fica no brilho do aco\\
            Do berimbau que eu toquei\\
            Berimbau deu um pulo no tempo\\
            Me encontrou nas profundezas\\
            Me deu toda a harmaonia\\
            No canto da Capoeira\\
            \textit{Sinto um vazio no peito...}\\
            Pensamento invade o passado\\
        \end{verse*}
\end{song}

\columnbreak

\begin{song}{title={Procurei um amigo }}
        \begin{verse*}
            Procurei um amigo\\
            \textit{Não veio}\\
            Fiquei na sauade\\
            \textit{Sozinho}\\
            Me pediu amizade\\
            \textit{Eu dei}\\
            Quem faz uma, faz duas\\
            \textit{Faz três}\\
            Quem faz Quatro, faz cinco\\
            \textit{Faz seis}\\
            Procurei um amigo\\
            \textit{Não veio}\\
            Fiquei na saudad\\
            \textit{Sozinho}\\
            Quem faz uma, faz dua\\
            \textit{Faz três}\\
            Quem faz Quatro, faz cinco\\
            \textit{Faz seis}\\
        \end{verse*}
\end{song}
\columnbreak
\begin{song}{title={Pimenta madura que dá semente}}
        \begin{verse*}
            Pimenta madura que dá semente\\
            Ô da semente, da semente\\
            \textit{Pimenta madura que dá semente}\\
            Pimenta madura que da semente\\
            \textit{Pimenta madura que dá semente}\\
            Que da semente, que da semente\\
            \textit{Pimenta madura que dá semente}\\
            Oi da semente, da semente\\
            \textit{Pimenta madura que dá semente}\\

        \end{verse*}
        \begin{verse*}
     Ingas Strophe?
        \end{verse*}
\end{song}


\chapTOC{Q}

\begin{song}{title={Que bom eu estar com você}}
        \begin{verse*}
            Que bom eu estar com você\\
            Aqui nessa roda\\
            Com esse conjunto\\
            axé pro meu mestre\\
            pro meu mestre axé\\
            e o vento que bate tão lindo\\
            que bate tão lindo\\
            Was singt der Chor mit?\\
        \end{verse*}
\end{song}


\begin{song}{title={Quem nunca andou de canoa}}
        \begin{verse*}
            \chord*{IV}Que m nunca andou de canoa,\\
            nao sabe o que e navegar\\
            Quem \chord*{V}nu nca jogou Capoeira de Angola\\
            nao sabe o que e vadiar\\
            \textit{(Quem nunca andou de canoa,}\\
            \textit{nao sabe o que e navegar}\\
            \textit{Quem nunca jogou Capoeira de Angola}\\
            \textit{nao sabe o que e vadiar)}\\
        \end{verse*}
\end{song}

\begin{song}{title={Quem vem lá? Sou eu}}
        \begin{verse*}
            \chord*{III}Qu em vem lá? Sou eu\\
            Quem vem lá? Sou eu\\
            Berimbau bateu\\
            Capoeira sou eu\\
            \textit{(Quem vem lá? Sou eu}\\
            \textit{Quem vem lá? Sou eu}\\
            \textit{Berimbau bateu}\\
            \textit{Capoeira sou eu)}\\
            Eu venho de longe\\
            Venho da Bahia\\
            Jogue capoeira\\
            Capoeira sou eu\\
            \textit{Coro}\\
            \chord*{I}E sou eu, sou es! (Einsatz sicher?)\\
            \chord*{I}\textit{Qu} \textit{em vem lá?}\\
            Sou eu Brevenuto\\
            \textit{Quem vem lá?}\\
            Montado a cavalo\\
            \textit{Quem vem lá?}\\
            Fumando charuto\\
            \textit{Quem vem lá?}\\
        \end{verse*}
\end{song},

\chapTOC{R}

\begin{song}{title={Rede, iô, iô}}
    \begin{verse*}
        \chord*{I}Re de, Iô, Iô\\
        Rede, iâ, iâ\\
        Capoeira, de, Angola joga\\
        Do, na, beira do mar\\
        \textit{Rede, Iô, Iô}\\
        \textit{Rede, iâ, iâ}\\
        \textit{Capoeira, de, Angola joga}\\
        \textit{Do, na, beira do mar}\\
    \end{verse*}
\end{song}

\chapTOC{S}

\begin{song}{title={Sai sai catarina}}
        \begin{verse*}
            Sai, sai, sai catarina\\
            Saia, do, mato venha ver Idalina\\
            \textit{Sai, sai, Catarina}\\
            E Catarina, venha, ver\\
            \textit{Sai, sai, Catarina}\\
            Vatarina, meu, amor\\
            \textit{Sai, sai, Catarina}\\
            Saia, do, mato venha ver, venha ver\\
            \textit{Sai, sai, Catarina}\\
            É catarina, minha, nega\\
            \textit{Sai, sai, Catarina}\\
            Mas,, oh, que saudade danada\\
            \textit{Sai, sai, Catarina}\\
            Dou, um, nó e escondo a ponta\\
            \textit{Sai, sai, Catarina}\\
            Prá, outro, desatar\\
            \textit{Sai, sai, Catarina}\\
        \end{verse*}
\end{song}

\begin{song}{title={Samba, lele}}
        \begin{verse*}
        \chord*{III}Sa mba lê lê bateu na porta\\
        Samba, lê, lê vai ver quem é\\
        Samba, lê, lê e meu amor\\
        Samba, lê, lê samba no pé\\
        \textit{Samba, lê, lê bateu na porta}\\
        \textit{Samba, lê, lê vai ver quem é}\\
        \textit{Samba, lê, lê e meu amor}\\
        \textit{Samba, lê, lê samba no pé}\\
    \end{verse*}

    \begin{verse*}
        Andam, dizendo, por aí\\
        Quem, o, amor se acabou\\
        E mentira, de, quem disse\\
        Ele, apenas, começou\\
        \textit{Coro},\\
        Mais, mulher velha não se casa\\
        Porque, não, acha ninguém\\
        Quando, chega, no roçado\\
        Chama, todos, de meu bem\\
\\
        \textit{Coro},\\
        Olha, aí moça bonita\\
        Coisa, linda, de se ver\\
        Você, é, o pé da rosa\\
        Eu, sou, a rosa do pé\\
        \end{verse*}
\end{song}

\begin{song}{title={Sant, Antonio, eu quero agua}}
        \begin{verse*}
            \chord*{III}Sa n, Antonio, quero agua\\
            \textit{Sant, Antonio, quero agua}\\
            \chord*{III}Que ro, Agua, pra beber,\\
            quero, agua, pra lavar,\\
            quero, agua, pra bencer\\
            \chord*{I}E u, quero agua\\
            \textit{Quero, Agua, pra beber...}\\
        \end{verse*}
\end{song}

\begin{song}{title={Sou, teu, irmão de roda}}
        \begin{verse*}
            \chord*{V?}So u, teu, irmão de roda\\
            \chord*{IV?}Te u, irmão, de capoeira\\
            Amizade, a, todas horas\\
            Amizade, sem, fronteira\\
        \end{verse*}
\end{song}

\chapTOC{T}

\begin{song}{title={Tem Areia}}
        \begin{verse*}
            \chord*{I}Tem, areia\\
            tem areia,\\
            tem, areia,\\
            \chord*{V}n o fundo do mar, tem areia\\
        \end{verse*}
\end{song}

\begin{song}{title={Tem, energia, na terra}}
        \begin{verse*}
            \chord*{V}Te m, energia, na terra\\
            Tem, energia, no mar\\
            A terra, e, a agua se encontram\\
            \chord*{IV}Fa zem, o, eixo da terra girar\\
            \textit{Tem, energia, na terra}\\
            \textit{Tem, energia, no mar}\\
            \textit{A, terra, e a agua se encontram}\\
            \textit{Faz, o, eixo da terra girar}\\
        \end{verse*}
\end{song}

\chapTOC{U}

\chapTOC{V}

\begin{song}{title={Vem, jogar, mais eu}}
        \begin{verse*}
            \chord*{V}Ve m, jogar, mais eu,\\
            Vem, jogar, mais eu mano meu,\\
            Oi, vem, jogar mais eu\\
            Mano, meu, vem jogar\\
        \end{verse*}
\end{song}

\begin{song}{title={Vim, la, da Bahia pra lhe ver}}
        \begin{verse*}
            \chord*{V}Vi m, la, da Bahia pra lhe ver
            Vim, la, da Bahia pra lhe ver\\
            Vim, la, da Bahia pra lhe ver,\\
            pra, lhe, ver, pra lhe ver, pra lhe ver, pra lhe ver\\
            \textit{Vim, la, da Bahia pra lhe ver...}\\
            \chord*{IV}Pr a, lhe, veeer, pra lhe veeeer, pra lhe ver, pra lhe ver, pra lhe ver\\
            \textit{Pra, lhe, veeer...}\\
        \end{verse*}
\end{song}

\begin{song}{title={Vivo, no ninho de cobra}}
        \begin{verse*}
            Sou, cobra, que cobra não morde\\
            Uma, cobra, conhece outra cobra\\
            \chord*{III?}Nã o, precisa, dizer quem é cobra\\
        \end{verse*}
\end{song}

\begin{song}{title={Vou, dizer a meu senhôr}}
        \begin{verse*}
            \chord*{III}Vo u, dizer, a meu senhor\\
            A manteiga, não, é minha\\
            \textit{Vou, dizer, a meu senhôr}\\
            \textit{Que, a, manteiga derramou}\\
            A, manteiga, não é minha\\
            A manteiga, é, de Ioiô\\
            \textit{Vou, dizer, a meu senhôr}\\
            \textit{Que, a, manteiga derramou}\\
\\
            A, manteiga, é de Ioiô\\
            Caiu, n água, e se molhou\\
            \textbf{(Chorus...)}\\
\\
            A, manteiga é do patrão\\
            Caiu, no, chão e derramou\\
            \textbf{(Chorus...)}\\
\\
            A, manteiga não é minha\\
            É pra, filha, de Ioiô\\
            \textbf{(Chorus...)}\\
        \end{verse*}
\end{song}

\columnbreak
\begin{song}{title={Vou, mandar, lecô cajuê }, subtitle={\begin{small}
            \url{https://www.youtube.com/watch?v=mv8Jr-BHGzg}
\end{small}}}
        \begin{verse*}
(Entweder, olho..., oder Vou mandar singen. Normalerweise gehts dann
direkt, weiter, mit "mandar le co")\\
\chord*{III}Ol ho, para, um lado vejo um capoeira\\
olho, para, o outro capoeira é\\
(\chord*{I}Vo u), mandar, lecô cajuê\\
(Vou), mandar, loiá\\
(Vou), mandar, lecô cajuê\\
(Vou), mandar, loiá\\
\textit{Vou, mandar, lecô cajuê}\\
\textit{Vou, mandar, loiá}\\
\textit{Vou, mandar, lecô cajuê}\\
\textit{Vou, mandar, loiá}\\
\chord*{I}Vo u, mandar, lecô\\
\chord*{III}\textit{Ca} \textit{juê}\\
Vou, mandar loiá\\
\textit{Cajuê},\\
Vou, mandar lecô\\
\textit{Cajuê},\\
Vou, mandar loiá\\
\textit{Cajuê},\\
        \end{verse*}
\end{song}

\chapTOC{W}
\chapTOC{X}
\chapTOC{Y}
\chapTOC{Z}

\chapTOC{Lieder von Mestre Índio}

\chapTOC{Ladainhas}


\begin{song}{title={Adeus, casa de farinha}}
        \begin{verse*}
terreiro de, ananás vou me embora pra \\ 
Bahia aqui eu não volto mais... \\
Iê, viva, meu Deus \\
Iê Galo Cantou\\
Iê cocoroco \\
Iê volta de Mundo \\
Iê que o mundo, deu! \\
        \end{verse*}
\end{song}

\begin{song}{title={Maior é Deus (Ladainha)}}
    \begin{verse*}
Iê\,\\
Maior, é Deus\\
Maior, é, Deus\\
Pequeno, sou, eu\\
O que, eu, tenho\\
foi, Deus, que me deu\\
O que, eu, tenho\\
foi, Deus, que me deu\\
Na, roda, da capoeira\\
Grande, e, pequeno sou eu (Haha)\\
Camará,\\
    \end{verse*}
\end{song}

\chapter*{Credits:}
[1] Jeder  gestorbene (Angola-)Meister kann hier eingesetzt werden.

\end{multicols*},
\end{document},
  
